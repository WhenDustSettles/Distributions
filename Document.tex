\documentclass[12pt, a4paper, titlepage]{article}

%\usepackage{tikz}
\usepackage{amssymb}
\usepackage{amsmath}
% font size could be 10pt (default), 11pt or 12 pt
% paper size coulde be letterpaper (default), legalpaper, executivepaper,
% a4paper, a5paper or b5paper
% side coulde be oneside (default) or twoside 
% columns coulde be onecolumn (default) or twocolumn
% graphics coulde be final (default) or draft 
%
% titlepage coulde be notitlepage (default) or titlepage which 
% makes an extra page for title 
% 
% paper alignment coulde be portrait (default) or landscape 
%
% equations coulde be 
%   default number of the equation on the rigth and equation centered 
%   leqno number on the left and equation centered 
%   fleqn number on the rigth and  equation on the left side
%	
\title{An Exhaustive (hopefully) list of Probability Distributions}
\author{Animesh Renanse \\
	Indian Institute of Technology Guwahati \\
	}

\date{\today} 
% \date{\today} date coulde be today 
% \date{25.12.00} or be a certain date
% \date{ } or there is no date 
\begin{document}
% Hint: \title{what ever}, \author{who care} and \date{when ever} could stand 
% before or after the \begin{document} command 
% BUT the \maketitle command MUST come AFTER the \begin{document} command! 
\maketitle
\newpage

%Each Distribution should have the following

%It's Plot (2D/3D depending on the nature)
%Notation 
%Parameters
%Support
%PDF
%CDF & it's derivation from PDF
%Mean(Expectation) & it's derivation
%Variance & it's derivation
%Moment Generating Function & it's derivation
%Entropy

\section{Basic Definitions}
This section provides the basic definitions of various statistical quantities which a particular probability distribution can have.

\begin{enumerate}

	\item{\textbf{Probability Density/Mass Function} : A PDF(PMF, in case of disrete random variables) describes the distribution of the likelihood of the outcome of a Random Variable over it's sample space or support.}

	\item{\textbf{Cumulative Distribution Function} : A CDF describes the probability of the outcome of a Random Variable falling less than a particular value.\\\\Consider A Continuous Random variable $X$, let it's PDF be denoted by $P(.)$, then it's CDF can be written as: 
\begin{center}$F_X(x)  = P(X \le x)$\end{center}

 Some properties of CDF are: 
\begin{enumerate}

	\item{$F_X(x)$ is non-decreasing and hence only has jump discontinuities.}
	\item{$\lim_{x \to \infty} F_X(x) = 1 $ and $\lim_{x \to -\infty} F_X(x) = 0 $}
	\item{CDF is right continuous, which means that: 
		\begin{center} $\lim_{h \to 0}F_X(x + h) = F_X(x)$  $\forall x \in \mathbb{R}$ \end{center}}
	\item{However, CDF is not left continous, but is related to left $\epsilon$ neighborhood by the following:
		\begin{center} $\lim_{h \to 0}F_X(x - h) = F_X(x) - P(X = x)$ $\forall x \in \mathbb{R}$\end{center}}

\end{enumerate}

}

	\item{\textbf{Expected Value} : This is the probability weighted average of a given random variable $X$. In physical sense, it represents the mean of a large number of independent results of the $X$.

Mathematically, consider a Random Variable $X$ whose support is the set $S_X$, then, two cases arise:
\begin{enumerate}

	\item{If X is Discrete, then: \begin{center} $$E\left[g(X)\right] = \sum_{x \in S_X} g(x) f_X(x) $$ \end{center}
			provided,  \begin{center} $$\sum_{x \in S_X} g(x) f_X(x) < \infty$$ \end{center}
			where $f_X(x)$ is the PMF of DRV $X$.}\\

	\item{If X is Continuous, then: \begin{center} $$E\left[g(X)\right] = \int_{x \in S_X} g(x) f_X(x) dx $$ \end{center}
			provided,  \begin{center} $$\int_{x \in S_X} g(x) f_X(x) dx< \infty$$ \end{center}
			}
	

\end{enumerate}
	
}


\end{enumerate}

\section{Probability Distributions}


%It's Plot (2D/3D depending on the nature)
%Notation 
%Parameters
%Support
%PDF
%CDF & it's derivation from PDF
%Mean(Expectation) & it's derivation
%Variance & it's derivation
%Moment Generating Function & it's derivation

\subsection{The Uniform Distribution}
The distribution describes an experiment where there is an arbitrary outcome that lies between certain bounds $a$ and $b$.\\
\begin{enumerate}
	%\item{Insert Image Here}%
	\item{Notation : $U(a,b)$ or $Unif(a,b)$}
	\item{Parameters : $-\infty < a < b < \infty$}
	\item{Support : $x \in [a,b]$}
	\item{Probability Distribution Function $f_X(.)$ :  \[   
f_X(x) = 
     \begin{cases}
       \text{$\frac{1}{b-a}$,} &\quad \text{$x \in [a,b]$}\\
       \text{0,} &\quad \text{otherwise}\\
     \end{cases}
\]}
	\item{Cumulative Distribution Function $F_X(.)$ : \\\\ For CDF, one needs to calculate cumulative probability coming up to this point, as shown below:-\\\newpage 

			\begin{center}

					$F_X(x) = \int_{-\infty}^{\infty} f_X(x) dx $

					$F_X(x) = \int_{a}^{b} \frac{1}{b-a} dx $

					$F_X(x) = \begin{cases}
	\text{0,} &\quad \text{$x < a$}\\
       \text{$\frac{x}{b-a}$,} &\quad \text{$x \in [a,b]$}\\
	\text{1,} &\quad \text{$x > b$}
     \end{cases}$	
			\end{center}

			 }
	\item{Expected Value ($\mu$): $\mathbb{E}_X(x) = \mu$ is the probability weighted average of all possible values of Random Variable $X$. It is calculated as shown below:-
\begin{center}
			\begin{align*}
					\mathbb{E}_X(x) &= \int_{-\infty}^{\infty} x f_X(x) dx 
					\\
					 &= \int_{a}^{b} x \frac{1}{b-a} dx 
					\\
					 &= \frac{1}{b-a} \int_{a}^{b} x dx 
					\\
					&= \frac{1}{b-a} \left[ \frac{x^2}{2} \right]_a^b
					\\
					&= \frac{b^2 - a^2}{2(b-a)}
					\\
					&= \frac{a+b}{2}
			\end{align*}
\end{center}
}\newpage
	\item{Variance ($\sigma$): This is just the Expected Value of $(X - \mathbb{E}(X))^2$.
\begin{center}
			\begin{align*}
					\sigma^2 &= \mathbb{E}\left( (X - \mathbb{E}(X))^2 \right)
					\\
					 &= \mathbb{E}\left( X^2 + \mathbb{E}(X)^2 - 2X\mathbb{E}(X) \right)
					\\
					 &= \mathbb{E}( X^2 )+ \mathbb{E}( \mathbb{E}(X)^2 ) -  \mathbb{E}(2X\mathbb{E}(X) )
					\\
					&= \frac{b^3 - a^3}{3(b-a)} + \frac{(a+b)^2}{4} - 2\frac{a+b}{2}\left(\frac{a+b}{2}\right)
					\\
					&= \frac{(b - a)^2}{12}
			\end{align*}
\end{center}


}
	\item{Moment Generating Function ($M_X(t)$): This is the expected value of $e^{tX}$
\begin{center}
			\begin{align*}
				MGF(X) = \mathbb{E}\left( e^{tX} \right) &= \int_{-\infty}^{\infty} e^{tx} f_X(x) dx
					\\
					 &= \frac{1}{b-a} \int_a^be^{tx} dx
					\\
					 &= \frac{1}{b-a} \left[ \frac{e^{tx}}{t} \right]_a^b
					\\
					&= \frac{e^{tb} - e^{ta}}{t(b-a)}
			\end{align*}
\end{center}




}
\end{enumerate}



%\tableofcontents % create a table of contens 

\end{document}